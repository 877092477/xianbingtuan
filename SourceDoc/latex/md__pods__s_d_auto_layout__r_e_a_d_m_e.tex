\subsection*{众多公司和个人开发者已经使用本库布局:}

S\+D\+Auto\+Layout使用者开发的部分app截图 \href{http://www.jianshu.com/p/9bc04d3effb8}{\tt http\+://www.\+jianshu.\+com/p/9bc04d3effb8}

一行代码搞定自动布局!致力于做最简单易用的\+Autolayout库。\+The most easy way for autolayout.

\subsection*{技术支持(Q\+Q交流群):}

497140713(1群) 519489682(2群已满)

\subsection*{Pod支持:}

支持pod: pod \textquotesingle{}S\+D\+Auto\+Layout\textquotesingle{}, \textquotesingle{}$\sim$$>$ 2.\+1.\+1\textquotesingle{}

如果发现pod search S\+D\+Auto\+Layout 搜索出来的不是最新版本,需要在终端执行cd转换文件路径命令退回到desktop,然后执行pod setup命令更新本地spec缓存(可能需要几分钟),然后再搜索就可以了

\subsection*{更新记录:}

2016.\+05.\+16 -- 修复用xib生成的view出现的部分约束失效问题(发布pod2.0.\+0版本)

2016.\+05.\+15 -- 增加设置偏移量offset功能

2016.\+04.\+30 -- 修复之前button作为父视图时内部控件不能自动布局问题

2016.\+04.\+05 -- 修复宽度自适应label在重用时候偶尔出现的宽度计算不准确的问题(发布pod1.51版本)

2016.\+03.\+23 -- 升级了缓存机制,新版本在tableview滑动cell时候流畅度和性能提升20以上(发布pod1.50版本)

2016.\+01.\+23 -- 增加label对attributed\+String的内容自适应

2016.\+01.\+21 -- 实现tableview局部刷新cell高度缓存的自动管理

2016.\+01.\+20 -- demo适配在ios7上的屏幕旋转问题

2016.\+01.\+18 -- 推出“普通简化版”tableview的cell自动高度方法(推荐使用),原来的需2步设置的普通版方法将标记过期

2016.\+01.\+13 -- 增加在不确定bottom view的情况下的cell高度自适应方法

2016.\+01.\+07 -- 1.增加 scrollview 横向内容自适应功能;2.\+增加view宽高相等的功能

2016.\+01.\+03 -- 增加任何类型对象都可以实现一行代码搞定cell高度自适应;增加文档注释

2015.\+12.\+08 -- 重大升级:1.\+支持scrollview内容自适应;2.\+任意添加或者修改约束不冲突;3.\+性能提升40以上;4.\+添加最大、最小宽高约束

\subsection*{视频教程:}

☆☆ S\+D\+Auto\+Layout 基础版视频教程:http\+://www.letv.\+com/ptv/vplay/24038772.html ☆☆

☆☆ S\+D\+Auto\+Layout 进阶版视频教程:http\+://www.letv.\+com/ptv/vplay/24381390.html ☆☆

☆☆ S\+D\+Auto\+Layout 原理简介视频教程:http\+://www.iqiyi.\+com/w\+\_\+19rt0tec4p.html ☆☆

\subsection*{部分\+S\+D\+Auto\+Layout的\+D\+E\+M\+O:}

完整微信\+Demo \href{https://github.com/gsdios/GSD_WeiXin}{\tt https\+://github.\+com/gsdios/\+G\+S\+D\+\_\+\+Wei\+Xin}



\section*{用法简介}

\subsection*{tableview和cell高度自适应:}

\paragraph*{普通(简化)版【推荐使用】:tableview 高度自适应设置只需要2步}

\begin{DoxyVerb}1. >> 设置cell高度自适应:
// cell布局设置好之后调用此方法就可以实现高度自适应(注意:如果用高度自适应则不要再以cell的底边为参照去布局其子view)
[cell setupAutoHeightWithBottomView:_view4 bottomMargin:10];

2. >> 获取自动计算出的cell高度

- (CGFloat)tableView:(UITableView *)tableView heightForRowAtIndexPath:(NSIndexPath *)indexPath
{
    id model = self.modelsArray[indexPath.row];
    // 获取cell高度
    return [self.tableView cellHeightForIndexPath:indexPath model:model keyPath:@"model" cellClass:[DemoVC9Cell class]  contentViewWidth:cellContentViewWith];
}
\end{DoxyVerb}


\paragraph*{升级版(适应于cell条数少于100的tableview):tableview 高度自适应设置只需要2步}

\begin{DoxyVerb}1. >> 设置cell高度自适应:
// cell布局设置好之后调用此方法就可以实现高度自适应(注意:如果用高度自适应则不要再以cell的底边为参照去布局其子view)
[cell setupAutoHeightWithBottomView:_view4 bottomMargin:10];

2. >> 获取自动计算出的cell高度 

- (CGFloat)tableView:(UITableView *)tableView heightForRowAtIndexPath:(NSIndexPath *)indexPath
{
// 获取cell高度
return [self cellHeightForIndexPath:indexPath cellContentViewWidth:[UIScreen mainScreen].bounds.size.width];
}
\end{DoxyVerb}


\subsection*{普通view的自动布局:}

\paragraph*{用法示例}

/$\ast$ 用法一 $\ast$/ \+\_\+view.\+sd\+\_\+layout .left\+Space\+To\+View(self.\+view, 10) .top\+Space\+To\+View(self.\+view, 80) .height\+Is(130) .width\+Ratio\+To\+View(self.\+view, 0.\+4); ~\newline
 /$\ast$ 用法二 (一行代码搞定,其实用法一也是一行代码) $\ast$/ \+\_\+view.\+sd\+\_\+layout.\+left\+Space\+To\+View(self.\+view, 10).top\+Space\+To\+View(self.\+view,80).height\+Is(130).width\+Ratio\+To\+View(self.\+view, 0.\+4);

\begin{DoxyVerb}>> UILabel文字自适应:
// autoHeightRatio() 传0则根据文字自动计算高度(传大于0的值则根据此数值设置高度和宽度的比值)
_label.sd_layout.autoHeightRatio(0);

*******************************************************************************

    注意:先把需要自动布局的view加入父view然后在进行自动布局,例: 

    UIView *view0 = [UIView new];
    UIView *view1 = [UIView new];
    [self.view addSubview:view0];
    [self.view addSubview:view1];

    view0.sd_layout
    .leftSpaceToView(self.view, 10)
    .topSpaceToView(self.view, 80)
    .heightIs(100)
    .widthRatioToView(self.view, 0.4);

    view1.sd_layout
    .leftSpaceToView(view0, 10)
    .topEqualToView(view0)
    .heightRatioToView(view0, 1)
    .rightSpaceToView(self.view, 10);
*******************************************************************************
\end{DoxyVerb}


\paragraph*{自动布局用法简析}



1.\+1 $>$ left\+Space\+To\+View(self.\+view, 10)

方法名中带有“\+Space\+To\+View”的方法表示到某个参照view的间距,需要传递2个参数:(\+U\+I\+View)参照view 和 (\+C\+G\+Float)间距数值

1.\+2 $>$ width\+Ratio\+To\+View(self.\+view, 1)

方法名中带有“\+Ratio\+To\+View”的方法表示view的宽度或者高度等属性相对于参照view的对应属性值的比例,需要传递2个参数:(\+U\+I\+View)参照view 和 (\+C\+G\+Float)倍数

1.\+3 $>$ top\+Equal\+To\+View(view)

方法名中带有“\+Equal\+To\+View”的方法表示view的某一属性等于参照view的对应的属性值,需要传递1个参数:(\+U\+I\+View)参照view

1.\+4 $>$ width\+Is(100)

方法名中带有“\+Is”的方法表示view的某一属性值等于参数数值,需要传递1个参数:(\+C\+G\+Float)数值

\section*{PS}

// 如果需要用“断言”调试程序请打开此宏(位于\+U\+I\+View+\+S\+D\+Auto\+Layout.h)

//\#define S\+D\+Debug\+With\+Assert

 